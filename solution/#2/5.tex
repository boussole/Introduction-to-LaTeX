\documentclass[a4paper,12pt]{article}

%%% Работа с русским языком
\usepackage{cmap}					% поиск в PDF
\usepackage{mathtext} 				% русские буквы в формулах
\usepackage[T2A]{fontenc}			% кодировка
\usepackage[utf8]{inputenc}			% кодировка исходного текста
\usepackage[english,russian]{babel}	% локализация и переносы

%%% Дополнительная работа с математикой
\usepackage{amsmath,amsfonts,amssymb,amsthm,mathtools} % AMS
\usepackage{icomma} % "Умная" запятая: $0,2$ --- число, $0, 2$ --- перечисление

%\usepackage{tikz, pgfplots}
%\usetikzlibrary{arrows}

\usepackage{tikz} % Работа с графикой
\usepackage{pgfplots}
\usepackage{pgfplotstable}
\usepgfplotslibrary{dateplot}

\begin{document}

\begin{center}

\begin{tikzpicture}
\begin{axis}[ %
	title={Динамика цен на нефть, USD/баррель},
	% горизонталь х - даты в российском формате повернуты на 90 градусов
	date coordinates in=x, %
	xticklabel style={rotate=90}, %
	xticklabel={\day.\month.\year}, %
	xmin=2015-10-01, %
	xmax=2015-12-31, %
	xtick={2015-10-01,2015-10-15,2015-11-01,2015-11-15,2015-12-01,2015-12-15,2015-12-31}, %
	line width=0.5mm
]
% строим из одного файла два разных графика
\addplot[blue] table[x=date,y=brent,col sep=comma]{5.csv};
\addplot[red] table[x=date,y=wti,col sep=comma]{5.csv};

\legend{Нефть Brent, Нефть WTI}

\end{axis}

\end{tikzpicture}

\end{center}

\end{document}