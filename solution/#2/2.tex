\documentclass[a4paper,12pt]{article}

%%% Работа с русским языком
\usepackage{cmap}                                       % поиск в PDF
\usepackage{mathtext}                           % русские буквы в формулах
\usepackage[T2A]{fontenc}                       % кодировка
\usepackage[utf8]{inputenc}                     % кодировка исходного текста
\usepackage[english,russian]{babel}     % локализация и переносы

%%% Дополнительная работа с математикой
\usepackage{amsfonts,amssymb,amsthm,mathtools} % AMS
\usepackage{amsmath}

%% Шрифты
\usepackage{euscript}    % Шрифт Евклид
\usepackage{mathrsfs} % Красивый матшрифт

% Окружение для уравнений с кастомным счетчиком
\newenvironment{myequation}[1]
{
	\renewcommand\theequation{#1}
	\multline
}
{
	\endmultline
}

\begin{document}

\begin{myequation}{$\Upsilon$}
b_2^{МНК}=
	\frac
	{
	\sum\limits_{i=1}^{n}(p_i-\overline{p})(w_i-\overline{w})
	}
	{
	\sum\limits_{i=1}^{n}(w_i-\overline{w})^2
	}
	=
	\frac
	{
	\sum\limits_{i=1}^{n}
		(\left[\beta_1+\beta_2w_i+u_{pi}\right]-
		\left[\beta_1+\beta_2\overline{w}+\overline{u}_p\right])(w_i-\overline{w})
	}
	{
	\sum\limits_{i=1}^{n}(w_i-\overline{w})^2
	}
	=
	\\
	=
	\frac
	{
	\sum\limits_{i=1}^{n}
		(\beta_2(w_i-\overline{w})(w_i-\overline{w})+(u_{pi}-\overline{u}_p)(w_i-\overline{w}))
	}
	{
	\sum\limits_{i=1}^{n}(w_i-\overline{w})^2
	}
	=
	\beta_2+
	\frac
	{
	\sum\limits_{i=1}^{n}(u_{pi}-\overline{u}_p)(w_i-\overline{w})
	}
	{
	\sum\limits_{i=1}^{n}(w_i-\overline{w})^2
	}.
\end{myequation}

\end{document} % Конец текста.
